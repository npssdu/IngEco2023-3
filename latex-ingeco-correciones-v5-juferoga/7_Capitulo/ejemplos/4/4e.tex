%%%%%%%%%%%%%%%%%%% EJERCICIO 4e %%%%%%
\textbf{e.}	Respuesta\\
	La tabla de Amortización será: \\
	\begin{spacing}{1.1}
		\begin{center}
			\begin{tabular}{|p{1cm}|p{2.5cm}|p{2.5cm}|p{2cm}|p{3cm}|}
				\hline
				\textbf{PER\ (1)} & \textbf{SALDO DEUDA (2)=(2)-(5)} & \textbf{INTERESES  (3)=(2)(i)} & \textbf{PAGO\ (4)=R  COP  -  L  COP} & \textbf{AMORTIZACIÓN  (5)=(4)-(3)} \\ \hline
				
				0                 &  2'000.000 COP                      & ---------                       & ---------                       & ---------                          \\ \hline
				1                 &  2'105.000  COP                      &  105.000  COP                   &  0,00  COP                    &    -105.000  COP                    \\ \hline
				2                 &  2'215.512  COP                      &  110.512  COP                   &  0,00  COP                    &    -110.512  COP                   \\ \hline
				3                 &  2'215.512  COP                      &  116.314  COP                  &  116.314  COP                  &    0,00  COP                           \\ \hline
				4                 &  2'215.512  COP                      &  116.314  COP                  &  116.314  COP                  &    0,00  COP                           \\ \hline
				5                 &  1'739.886  COP                       &  116.314  COP                  &  591.940  COP                 &    475.625,59   COP                    \\ \hline
				6                 &  1'239.290  COP                       &  91.344  COP                   &  591.940  COP                 &    500.596  COP                     \\ \hline
				7                 &  562.413  COP                        &   65.062  COP                   &  741.940  COP                 &    676.877   COP                   \\ \hline
				8                 &  0,00  COP                           &  29.526  COP                &    591.940  COP                   &    562.413   COP                    \\ \hline
			\end{tabular}
		\end{center}
	\end{spacing}
	\textbf{Observaciones: }\\
	1)	La amortización de los períodos 1 y 2 es negativa; esto significa que hay una desamortización o aumento de deuda.\\
	2)	La amortización de los períodos 3 y 4 es cero, debido a que se pagan los intereses.\\
	3)	 El valor del pago en el período 7 es igual a la suma del pago ordinario, más el pago extraordinario, por tanto:  591.940,1 
 COP  +  150.000  COP  =  741.940,1  COP.\\
%%%%%%%%%%%%%%%%%%%%%%%%%%FIN EJERCICIO 4e %%%%%%%%%%%%%%%%%%%%%%%%%%%