\textbf{Ejemplo 5}\\
Dada una tasa del 24\% nominal anual mes vencido, hallar una tasa nominal anual semestre vencido equivalente.\\

%%%%%%%%%%%%%%%%%%% EJERCICIO 1 %%%%%%

%\newpage %USAR SOLO SI EL SOLUCIÓN QUEDA SOLO Y ES NECESARIO BAJARLO A LA SIGUIENTE PAGINA
\textbf{Solución.}\\
%La tabla ira centrada
\begin{center}
  \renewcommand{\arraystretch}{1.5}% Margenes de las celdas
  %Creación de la cuadricula de 3 columnas \end{flushleft}
  \begin{longtable}[H]{|C{0.3\linewidth}|C{0.3\linewidth}|C{0.3\linewidth}|}
    %Creamos una linea horizontal
    \hline
    %%%%%%%%%% INICIO TITULO
    %Lo que se hace aquí es mezclar las 3 columnas en una sola
    \multicolumn{3}{|c|}{\cellcolor[HTML]{FFB183}\textbf{1. Asignación período focal}}                                      \\ \hline
    %%%%%%%%%% FIN TITULO
    %%%%% INICIO DECLARACIÓN DE VARIABLES %%%%%%%
    \multicolumn{3}{|c|}{No aplica}                                                                                         \\ \hline
    %%%%%%%%%% INICIO TITULO
    %Lo que se hace aquí es mezclar las 3 columnas en una sola
    \multicolumn{3}{|c|}{\cellcolor[HTML]{FFB183}\textbf{2. Declaración de variables}}                                      \\ \hline
    %%%%%%%%%% FIN TITULO
    %%%%%%%%%% INICIO DE MATEMÁTICAS
    %Cada & hace referencia al paso de la siguiente columna
    $j_{1}=24\%namv$ & $i_{1}=\frac{24\%namv}{12pmv}=2\%pmv$ & $j_{2}=?\%nasv$                                              \\
    $m_{1}=12pmv$    &                                       &                                                              \\
    $m_{2}=2psv$     &                                       &                                                              \\ \hline

    %%%%%%%%%% FIN DE MATEMÁTICAS
    %%%%% FIN DECLARACIÓN DE VARIABLES


    %%%%% INICIO FLUJO DE CAJA
    \rowcolor[HTML]{FFB183}
    \multicolumn{3}{|c|}{\cellcolor[HTML]{FFB183}\textbf{2. Diagrama de flujo de caja}}                                     \\ \hline
    %Mezclamos 3 columnas y pondremos el dibujo
    %%%%%%%%%%%%% INSERCIÓN DE LA IMAGEN
    %Deberán descargar las imágenes respectivas del drive y pegarlas en la carpeta
    %n_capitulo/img/ejemplos/1/capitulo1ejemplo1.pdf  (el /1/ es el numero del ejemplo)
    \multicolumn{3}{|c|}{ \includegraphics[trim=-5 -5 -5 -5 , scale=0.4]{2_Capitulo/img/ejemplos/6/Capitulo2Ejemplo6.pdf} } \\ \hline
    %%%%%%%%%%%%% FIN INSERCIÓN DE IMAGEN
    %%%%%FIN FLUJO DE CAJA



    %%%%% INICIO DECLARACIÓN FORMULAS
    %%%%%%%%%%% INICIO TITULO
    \rowcolor[HTML]{FFB183}
    \multicolumn{3}{|c|}{\cellcolor[HTML]{FFB183}\textbf{3. Declaración de fórmulas}}                                       \\ \hline
    %%%%%%%%%%% FIN TITULO
    %%%%%%%%%%% INICIO MATEMÁTICAS
    \multicolumn{3}{|c|}{$(1+i_{1})^{m_1} = (1+i_{2})^{m_2}\hspace{0.3cm} \textit{Equivalencia de tasas}$}                  \\
    \multicolumn{3}{|c|}{$j=im\hspace{0.3cm} \textit{Tasa periódica anualizada}$}
    \\ \hline
    %%%%%%%%%% FIN MATEMÁTICAS
    %%%%%% INICIO DESARROLLO MATEMÁTICO
    \rowcolor[HTML]{FFB183}
    %%%%%%%%%%INICIO TITULO
    \multicolumn{3}{|c|}{\cellcolor[HTML]{FFB183}\textbf{4. Desarrollo matemático}}                                         \\ \hline
    %%%%%%%%%% FIN TITULO
    %%%%%%%%%% INICIO MATEMÁTICAS
    \multicolumn{3}{|c|}{$(1+0.02)^{12}=(1+i_{2})^2$}                                                                       \\
    \multicolumn{3}{|c|}{$i_{2}=0.126162 \equiv 12.6162\%psv$}                                                                         \\
    \multicolumn{3}{|c|}{$j_{2}=(12.6162\%psv)(2psv)$}                                                                 \\
    \multicolumn{3}{|c|}{$j_{2}=25.2324\%nasv$}
    \\ \hline


    %%%%%%%%%% FIN MATEMÁTICAS
    %%%%%% FIN DESARROLLO MATEMÁTICO
    %%%%%% INICIO RESPUESTA
    \rowcolor[HTML]{FFB183}
    %%%%%%%%%%INICIO TITULO
    \multicolumn{3}{|c|}{\cellcolor[HTML]{FFB183}\textbf{5. Respuesta}}                                                     \\ \hline
    %%%%%%%%%% FIN TITULO
    %%%%%%%%%% INICIO RESPUESTA MATEMÁTICA
    \multicolumn{3}{|c|}{$j_{2}=25.2324\%nasv$}                                                                               \\ \hline
    %%%%%%%%%% FIN MATEMÁTICAS
    %%%%%% FIN RESPUESTA
  \end{longtable}
  %Se crean dos lineas en blanco para que no quede el siguiente texto tan pegado
  %\newline \newline %USARLO SI CREES QUE ES NECESARIO
\end{center}
%%%%%%%%%%%%%%%%%%%%%%%%%%FIN EJERCICIO 1 %%%%%%%%%%%%%%%%%%%%%%%%%%%
