\newpage
\textbf{Ejemplo 6}\\
Una industria tiene actualmente contratado un préstamo con una
corporación financiera a la tasa del TCC+3 puntos. ¿Cuál debe ser el spread
en puntos básicos de forma tal que financiera mente sea indiferente el
préstamo en la corporación financiera o en el mercado de Londres? con
estos índices:\\ \\
\begin{itemize}
 \item $TCC = 15,3\% \textit{ nata}$
 \item $i_{dev} = 22\% \textit{ naav}$
\end{itemize}
Interés devaluación del peso Colombiano con respecto a la libra esterlina Libor
\begin{itemize}
 \item  $Libor= 5,2\% \textit{ naav}$. Tasa de interés del mercado europeo
\end{itemize}
%\newpage %USAR SOLO SI EL SOLUCIÓN QUEDA SOLO Y ES NECESARIO BAJARLO A LA SIGUIENTE PAGINA
\textbf{Solución.}\\
%La tabla ira centrada
\begin{center}
 \renewcommand{\arraystretch}{1.5}% Margenes de las celdas
 %Creación de la cuadricula de 3 columnas
 \begin{longtable}[H]{|p{0.375\linewidth}|p{0.375\linewidth}|p{0.1\linewidth}|}
  %Creamos una linea horizontal
  \hline
  %Definimos el color de la primera fila
  \rowcolor[HTML]{FFB183}
  %%%%% INICIO ASIGNACIÓN FECHA FOCAL %%%%%%%
  %%%%%%%%%% INICIO TITULO
  %Lo que se hace aquí es mezclar las 3 columnas en una sola
  \multicolumn{3}{|c|}{\cellcolor[HTML]{FFB183}\textbf{1. Asignación período focal}}   \\ \hline
  %%%%%%%%%% FIN TITULO
  \multicolumn{3}{|c|}{$pf=1 \textit{ pav}$}                                         \\ \hline
  %%%%% INICIO DECLARACIÓN DE VARIABLES %%%%%%%
  %%%%%%%%%% INICIO TITULO
  %Lo que se hace aquí es mezclar las 3 columnas en una sola
  \multicolumn{3}{|c|}{\cellcolor[HTML]{FFB183}\textbf{2. Declaración de variables}} \\ \hline
  %%%%%%%%%% FIN TITULO
  %%%%%%%%%% INICIO DE MATEMÁTICAS
  %Cada & hace referencia al paso de la siguiente columna
  \multicolumn{2}{|C{0.75\linewidth}|}{$TCC = 15,3\% \textit{ nata}$}   & $X=?$      \\
  \multicolumn{2}{|C{0.75\linewidth}|}{$i_{dev} = 22\% \textit{ naav}$} &            \\
  \multicolumn{2}{|C{0.75\linewidth}|}{$Libor = 5,2\% \textit{ naav}$}  &            \\ \hline

  %%%%%%%%%% FIN DE MATEMÁTICAS
  %%%%% FIN DECLARACIÓN DE VARIABLE


  %%%%% INICIO DECLARACIÓN FORMULAS
  %%%%%%%%%%% INICIO TITULO
  \rowcolor[HTML]{FFB183}
  \multicolumn{3}{|c|}{\cellcolor[HTML]{FFB183}\textbf{3. Declaración de fórmulas}}  \\ \hline
  %%%%%%%%%%% FIN TITULO
  %%%%%%%%%%% INICIO MATEMÁTICAS

  \multicolumn{3}{|c|}{$i_{dev} + Libor + X = TCC + 3 \textit{ Ecuación de valor}$}  \\ \hline
  %%%%%%%%%% FIN MATEMÁTICAS
  %%%%%% INICIO DESARROLLO MATEMÁTICO
  \rowcolor[HTML]{FFB183}
  %%%%%%%%%%INICIO TITULO
  \multicolumn{3}{|c|}{\cellcolor[HTML]{FFB183}\textbf{4. Desarrollo matemático}}    \\ \hline
  %%%%%%%%%% FIN TITULO
  %%%%%%%%%% INICIO MATEMÁTICAS
  \multicolumn{3}{|p{\textwidth}|}{
  $TCC + 3 puntos = 15,3\% \hspace{1mm} nata + 3\% \hspace{1mm} nata = 18,3\% \hspace{1mm} nata$

  $j_{crédito} = TCC + 3 puntos = 15,3\% \hspace{1mm} nata + 3\% \hspace{1mm} nata = 18,3\% \hspace{1mm} nata$

  $i = i_1 + i_2 + i_1 \cdot i_2 $ Equivalencia de tasas de referencia

  $j_c = 18,3\% \hspace{1mm} nata$

  $j_c =? naav$ en Londres

  $18,3\% \hspace{1mm} nata = 20,601\%naav$

  \vspace{4mm}
  Teniendo la tasa periódica vencida utilizamos la fórmula de las tasas equivalentes
  \vspace{1mm}

  $(1 + i_1)^{m_1} = (1 + i_2)^{m_2}, donde \hspace{1mm} m_1 = 1 \hspace{1mm} pav$

  $(1 + i_1)^1 = (1 + 0,183)^4$

  $i_{dev} + Libor + X = 22\% + (5,2 + X)\%$

  Crédito equivalente = $22\% naav + (Libor + X)\%naav =? \%naav$

  \vspace{4mm}
  Por la formula de combinación de tasas, por ser tasas naav
  \vspace{1mm}
  $0,28344 + 0,0122 + X = 0,20601$ (Despejar la X)
  $X = -6,35\%$
  }                                                                                  \\ \hline
  %%%%%%%%%% FIN MATEMÁTICAS
  %%%%%% FIN DESARROLLO MATEMÁTICO
  %%%%%% INICIO RESPUESTA
  \rowcolor[HTML]{FFB183}
  %%%%%%%%%%INICIO TITULO
  \multicolumn{3}{|c|}{\cellcolor[HTML]{FFB183}\textbf{5. Respuesta}}                \\ \hline
  %%%%%%%%%% FIN TITULO
  %%%%%%%%%% INICIO RESPUESTA MATEMÁTICA
  \multicolumn{3}{|C{\textwidth}|}{
  X = -6,35\%
  }                                                                                  \\ \hline
  %%%%%%%%%% FIN MATEMÁTICAS
  %%%%%% FIN RESPUESTA
 \end{longtable}
 %Se crean dos lineas en blanco para que no quede el siguiente texto tan pegado
 %\newline \newline %USARLO SI CREES QUE ES NECESARIO
\end{center}