%----------------------------------------------------------------------------------------
% glosario
%----------------------------------------------------------------------------------------


\part{Glosario}
\graphicspath{ {W_Varios/2_Portada_capitulos}, {W_Varios/2_Portada_capitulos/} }

%----------------------------------------------------------------------------------------
%	Glosario
%----------------------------------------------------------------------------------------

\chapterimage{ima2} % Chapter heading image



\chapter{Glosario}

\begin{itemize}

	\item \textbf{Banco: }
	      Institución financiera de intermediación que recibe fondos en forma de depósito de las personas que poseen excedentes de liquidez, utilizándolos posteriormente para operaciones de préstamo a personas con necesidades de financiación, o para inversiones propias. Presta también servicios de todo tipo relacionados con cualquier actividad realizada en el marco de actuación de un sistema financiero.\\ \\
	      
	\item \textbf{Banco Comercial: }
	      Institución que se dedica al negocio de recibir dinero en depósito y darlo a su vez en préstamo, sea en forma de mutuo, de descuento de documentos o de cualquier otra forma. Se consideran además todas las operaciones que natural y legalmente constituyen el giro bancario.\\
	      
	\item \textbf{Banco de la república: }
	      Banco central y emisor de Colombia que por mandato constitucional tiene como función principal luchar contra la inflación.\\
	      
	\item \textbf{Banco de segundo piso:} Instituciones financieras que no tratan directamente con los usuarios de los créditos, sino que hacen las colocaciones de los mismos a través de otras instituciones financieras.\\
	      
	\item \textbf{Banco Interamericano del Desarrollo (BID): }
	      Su objetivo es financiar proyectos de desarrollo y asistencia mutua en las naciones del continente americano, además de promover el crecimiento económico y social de los países miembros.\\
	      
	\item \textbf{Banco Mundial (BM): }
	      Creado en 1945 con el propósito de reconstruir a Europa una vez terminada la Segunda Guerra Mundial, pero ahora se dedica a financiar proyectos de inversión para el desarrollo de los países miembros.\\
	      
	\item {\textbf{Deflación (if); }}
	      Disminución generalizada en el nivel de precios. Dicha situación casi siempre ocurre cuando la economía se encuentra en recesión.\\
	      
	\item {\textbf{Deflactor (def): }}
	      Índice que se utiliza para convertir un valor o precio corriente (nominal) a uno constante respecto a un precio de referencia.\\
	      
	\item {\textbf{Devaluación (idev): }}
	      La pérdida de valor de una moneda frente a otra moneda se denomina devaluación, por ejemplo habrá devaluación si inicialmente hay que pagar  COP 1500 por un dólar y un año más tarde hay que pagar  COP 2000 por el mismo dólar. En este caso la devaluación del año es igual a la variación de precio sobre el precio inicial, esto es:\\
	      
	      devaluación = $\frac{ COP 2000- COP 1500}{ COP 1500} = 0,333333 $ = 33,33\% período Anual Vencido \\
	      
	      Lo contrario de la devaluación se denomina revaluación que significa que habrá que pagar menos pesos por el mismo dólar, por ejemplo si al principio del año hay que pagar  COP 1500 por un dólar y al final del año hay que pagar  COP 1200 entonces la revaluación será variación de precio sobre el precio inicial así: \\
	      
	      revaluación = $\frac{ COP 1200- COP 1500}{ COP 1500} = -0,2$ = -20\% período Anual Vencido \\
	      
	      
	\item {\textbf{Estanflación (if): }}
	      período de tiempo donde se presenta una alta inflación acompañada de una contracción económica. En términos generales, un acelerado crecimiento económico puede inducir a un aumento en el nivel de precios.\\
	      
	\item \textbf{Fondo Monetario Internacional(FMI): }
	      Institución financiera internacional creada en 1946 con el fin de estabilizar el Sistema Monetario Internacional, que tiene como funciones principales vigilar las políticas referentes al tipo de cambio de los países miembros y prestar recursos para apoyar sus políticas de ajuste y programas de estabilización\\
	      
	\item {\textbf{Índice: }}
	      Es un indicador que tiene por objeto medir las variaciones de un fenómeno económico o de otro orden referido a un valor que se toma como base en un momento dado. Relación de precios, de cantidades, de valores entre dos períodos dados.\\
	      
	\item{\textbf{Índice de precios al productor (IPP): }}
	      Aunque es similar al índice de Precios al Consumidor (IPC), éste mide las variaciones que muestran los precios de bienes y servicios intermedios, es decir, de aquellos consumidos en el proceso de producción, tales como las materias primas.
	      
	\item{\textbf{Índice de precios al productor (IPP): }}
	      Aunque es similar al índice de Precios al Consumidor (IPC), éste mide las variaciones que muestran los precios de bienes y servicios intermedios, es decir, de aquellos consumidos en el proceso de producción, tales como las materias primas.\\
	      
	\item {\textbf{Índice de tasa de cambio real (ITCR): }}
	      Medida más amplia que la tasa de cambio de la moneda local en relación con una canasta de monedas de otros países y sus respectivas inflaciones o devaluaciones. Un Indice de tasa de cambio real (ITCR) que sea igual a cien (100), indica que la moneda local de un país no está devaluada ni revaluada en comparación con otras monedas internacionales. \\
	      
	\item {\textbf{Inflación (if)}: }
	      Mide el crecimiento del nivel general de precios de la economía. La inflación es calculada mensualmente por el DANE sobre los precios de una canasta básica de bienes y servicios de consumo para familias de ingresos medios y bajos. Con base en éstas calcula un índice denominado Índice de Precios al Consumidor (IPC). La inflación corresponde a la variación periódica de ese Índice.\\
	      
	      
	\item {\textbf{LIBOR (London Interbank Offer Rate): }}
	      Tasa de interés anual vigente para los préstamos interbancarios de primera clase en Londres y Europa.\\
	      
	\item {\textbf{Papel de renta fija: }}
	      Título valor representativo de una deuda que da a quien lo posee, el derecho a recibir un interés fijo por período preestablecido.
	      
	\item {\textbf{Papeles con descuento: }}
	      Títulos valores representativos de deuda, que no generan interés. Su rendimiento se obtiene de la diferencia entre su valor de adquisición y su valor nominal.
	      \\
	\item {\textbf{Papeles de renta variable: }}
	      Títulos valores que por sus características solo permiten conocer la rentabilidad de la inversión en el momento de su rendición, dependiendo de la entidad emisora entre otras. Ejemplo: Acciones.
	      \\
	\item {\textbf{Pagaré: }}
	      Una promesa incondicional que hace por escrito una persona a otra, firmada por el librador, de pagar a la vista o en una fecha defina futura, una suma determina de dinero a la orden de una persona especifica o un portador.
	      \\
	      Título donde consta el monto del dinero prestado, el interés, la garantía, forma de pago y otras características del crédito.
	      \\
	\item {\textbf{Pago en efectivo: }}
	      Transacción en la que un banco hace un pago inmediato en billetes o monedas.
	      \\
	\item {\textbf{Papel comercial: }}
	      Son pagares ofrecidos públicamente en el mercado de valor y emitidos por sociedades anónimas, limitadas y entidades públicas, no sujetos a inspección y vigilancia de la superintendencia financiera. Su vencimiento no puede ser inferior a 15 ni superior a 9 meses. La rentabilidad la determina el emisor de acuerdo con las condiciones del mercado.
	      \\
	\item {\textbf{Período de gracia: }}
	      El período transcurrido entre la fecha en la que se realiza la operación financiera y la fecha en que se da el primer pago.
	      
	\item {\textbf{Revaluación (irev): }}
	      Es la apreciación del peso frente a otra moneda de referencia. Es el concepto opuesto a la devaluación.\\
	      
	\item {\textbf{Tasa Depósito a Término Fijo (DTF): }}
	      Es un indicador que recoge el promedio semanal de la tasa de captación de los certificados de depósito a término (CDTs) a 90 días de los bancos y compañias de financiamiento comercial, es calculado por el Banco de la República. Hay para 180 y 360 días también. \\
	      
	      
	\item  {\textbf{Tasa promedio de Captaciones de las Corporaciones (TCC): }}
	      Es la tasa promedio de captación de los certificados de depósito a término de las corporaciones financieras y es calculada por el Banco de la República\\
	      
	\item {\textbf{Tasa promedio de Captaciones de las Corporaciones (TCC: )}}
	      Es la tasa promedio de captación de los certificados de depósito a término de las corporaciones financieras y es calculada por el Banco de la República\\
	      
\end{itemize}
