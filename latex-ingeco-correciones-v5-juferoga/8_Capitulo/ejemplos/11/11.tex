\textbf{Ejemplo 11}\\
Hallar el valor presente de 15 pagos que decrecen
linealmente en 400 COP, si el primer pago es de 5.000 COP y la tasa efectiva es del 4\% período año
vencido..\\ \\
%\newpage %USAR SOLO SI EL SOLUCIÓN QUEDA SOLO Y ES NECESARIO BAJARLO A LA SIGUIENTE PAGINA
\textbf{Solución.}\\
%La tabla ira centrada
\begin{center}
 \renewcommand{\arraystretch}{1.5}% Margenes de las celdas
 %Creación de la cuadricula de 3 columnas
 \begin{longtable}[H]{|p{0.5\linewidth}|p{0.5\linewidth}|}
  \hline
  \multicolumn{2}{|c|}{\cellcolor[HTML]{FFB183}\textbf{1. Declaración de variables}}                                                                                                                 \\ \hline
  $R =  5.400 COP$                                                                                & $L = 400 COP$                                                                                   \\
  $n=15 \hspace{1mm} pav$                                                                         & $VP=? COP$                                                                                         \\
  $i=4,0\% \hspace{1mm} pav$                                                                      &                                                                                                  \\
  \multicolumn{2}{|c|}{\cellcolor[HTML]{FFB183}\textbf{2. Diagrama de flujo de caja}}                                                                                                                \\ \hline
  \multicolumn{1}{|c|}{\includegraphics[trim=-5 -5 -5 -5 ,width=0.5\columnwidth]{11/Tabla 1.pdf}} & \multicolumn{1}{|c|}{ \includegraphics[trim=-5 -5 -5 -5 ,width=0.5\columnwidth]{11/Tabla 2.pdf}} \\ \hline
  \multicolumn{2}{|c|}{\cellcolor[HTML]{FFB183}\textbf{3. Aplicación de funciones}}                                                                                                                  \\ \hline
  \multicolumn{2}{|p{\columnwidth}|}{Se aplicará la función valor presente VNA de la siguiente forma: \newline
  =VNA(0,04;J8::J22) con referencia en la hoja de
  Excel usada para el ejercicio.}                                                                                                                                                                    \\
  \multicolumn{2}{|c|}{ \includegraphics[trim=-5 -5 -5 -5 ,width=0.7\columnwidth]{11/excel1.png}}                                                                                                    \\ \hline
  \multicolumn{2}{|c|}{\cellcolor[HTML]{FFB183}\textbf{4. Gráfica}}                                                                                                                                  \\ \hline
  \multicolumn{2}{|c|}{ \includegraphics[trim=-5 -5 -5 -5 ,width=0.8\columnwidth]{11/gráfico.pdf}}                                                                                                   \\ \hline
  \multicolumn{2}{|c|}{\cellcolor[HTML]{FFB183}\textbf{5. Respuesta}}                                                                                                                                \\ \hline
  \multicolumn{2}{|c|}{El valor presente es VP =  COP 27.697,9410}                                                                                                                                      \\ \hline
 \end{longtable}
 %\newline \newline %USARLO SI CREES QUE ES NECESARIO
\end{center}
