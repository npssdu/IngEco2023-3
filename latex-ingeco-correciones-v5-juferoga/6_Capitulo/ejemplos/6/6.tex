	
	\textbf{Ejemplo 6}\\
	Calcular el valor presente de una serie infinita de egresos que crecen en  10{.}000COP, si el primer egreso es de  200{.}000COP y la tasa es del 3\% periódica mes vencido.\\
	
	\newpage
	
	%%%%%%%%%%%%%%%%%%% EJERCICIO 6 %%%%%%
	
	%\newpage %USAR SOLO SI EL SOLUCIÓN QUEDA SOLO Y ES NECESARIO BAJARLO A LA SIGUIENTE PAGINA
	\textbf{Solución.}\\
	%La tabla ira centrada
	\begin{center}
		\renewcommand{\arraystretch}{1.6}% Margenes de las celdas
		%Creación de la cuadricula de 3 columnas
		\begin{longtable}[H]{|c|c|c|}
			%Creamos una linea horizontal
			\hline
			%Definimos el color de la primera fila
			\rowcolor[HTML]{FFB183}
			%%%%% INICIO ASIGNACIÓN FECHA FOCAL %%%%%%%
			%%%%%%%%%% INICIO TITULO
			%Lo que se hace aquí es mezclar las 3 columnas en una sola
			\multicolumn{3}{|c|}{\cellcolor[HTML]{FFB183}\textbf{1. Asignación período focal}}  \\ \hline
			\multicolumn{3}{|c|}{$pf = \textit{0 pmv}$}   \\\hline
			%%%%%%%%%% FIN TITULO
			%%%%% INICIO DECLARACIÓN DE VARIABLES %%%%%%%
			%%%%%%%%%% INICIO TITULO
			%Lo que se hace aquí es mezclar las 3 columnas en una sola
			\multicolumn{3}{|c|}{\cellcolor[HTML]{FFB183}\textbf{2. Declaración de variables}}   \\ \hline
			%%%%%%%%%% FIN TITULO
			%%%%%%%%%% INICIO DE MATEMÁTICAS
			%Cada & hace referencia al paso de la siguiente columna
			\multicolumn{2}{|c|}{\textbf{$\hspace{3.5 cm}\textit{}\hspace{3.5 cm}$}} & \textbf{$\hspace{3.5 cm}\textit{}\hspace{3.5 cm}$} \\ 
			\multicolumn{2}{|c|}{$\hspace{2 cm}L=  10{.}000COP \hspace{2 cm}$} & {$i=3\% \textit{ pmv}$} \\
			\multicolumn{2}{|c|}{$\hspace{2 cm}R=   200{.}000COP \hspace{2 cm}$} & $n=\infty \textit{ pmv}$ \\ 	
			\multicolumn{2}{|c|}{$\hspace{2cm} VP = ? COP \hspace{2 cm}$ } & $$\\ \hline
			%%%%%%%%%% FIN DE MATEMÁTICAS
			%%%%% FIN DECLARACIÓN DE VARIABLES
			
			%%%%% INICIO FLUJO DE CAJA
			\rowcolor[HTML]{FFB183}
			\multicolumn{3}{|c|}{\cellcolor[HTML]{FFB183}\textbf{3. Diagrama de flujo de caja}} \\ \hline
			%Mezclamos 3 columnas y pondremos el dibujo
			%%%%%%%%%%%%% INSERCIÓN DE LA IMAGEN
			%Deberán descargar las imágenes respectivas del drive y pegarlas en la carpeta
			%n_capitulo/img/ejemplos/1/capitulo1ejemplo1.pdf  (el /1/ es el numero del ejemplo)
			\multicolumn{3}{|c|}{ \includegraphics[trim=-5 -5 -5 -5 , scale=0.6]{6_Capitulo/img/ejemplos/6/capitulo6ejemplo6.pdf} }
			
			\\ \hline
			%%%%%%%%%%%%% FIN INSERCIÓN DE IMAGEN
			%%%%%FIN FLUJO DE CAJA
			
			%%%%% INICIO DECLARACIÓN FORMULAS
			%%%%%%%%%%% INICIO TITULO
			\rowcolor[HTML]{FFB183}
			\multicolumn{3}{|c|}{\cellcolor[HTML]{FFB183}\textbf{4. Declaración de fórmulas}}    \\ \hline
			%%%%%%%%%%% FIN TITULO
			%%%%%%%%%%% INICIO MATEMÁTICAS
			
			\multicolumn{3}{|c|}{$VP=(\frac{R}{i})+(\frac{L}{i^2}) \hspace{0.4 cm} \textit{Valor presente de un gradiente aritmetico}$} \\ \hline
			
			%%%%%%%%%% FIN MATEMÁTICAS
			%%%%%% INICIO DESARROLLO MATEMÁTICO
			\rowcolor[HTML]{FFB183}
			%%%%%%%%%%INICIO TITULO
			\multicolumn{3}{|c|}{\cellcolor[HTML]{FFB183}\textbf{5. Desarrollo matemático}}       \\ \hline
			%%%%%%%%%% FIN TITULO
			%%%%%%%%%% INICIO MATEMÁTICAS
			\multicolumn{3}{|c|}{$VP=(\frac{ 200{.}000}{0,03})+(\frac{  10{.}000COP}{0,03^2}) \hspace{0.2 cm}\rightarrow \hspace{0.2 cm}VP= COP 17{.}777{.}778$} \\ \hline
			
			%%%%%%%%%% FIN MATEMÁTICAS
			%%%%%% FIN DESARROLLO MATEMÁTICO
			%%%%%% INICIO RESPUESTA
			\rowcolor[HTML]{FFB183}
			%%%%%%%%%%INICIO TITULO
			\multicolumn{3}{|c|}{\cellcolor[HTML]{FFB183}\textbf{6. Respuesta}}   \\ \hline
			%%%%%%%%%% FIN TITULO
			%%%%%%%%%% INICIO RESPUESTA MATEMÁTICA
			\multicolumn{3}{|c|}{$ VP=  17{.}777.78 COP$} 
			\\ \hline
			%%%%%%%%%% FIN MATEMÁTICAS
			%%%%%% FIN RESPUESTA
		\end{longtable}
		%Se crean dos lineas en blanco para que no quede el siguiente texto tan pegado
		%\newline \newline %USARLO SI CREES QUE ES NECESARIO
	\end{center}
	%%%%%%%%%%%%%%%%%%%%%%%%%%FIN EJERCICIO 6 %%%%%%%%%%%%%%%%%%%%%%%%%%%
