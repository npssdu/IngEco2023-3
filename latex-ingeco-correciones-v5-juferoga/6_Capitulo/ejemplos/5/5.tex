\textbf{Ejemplo 5}\\
a. Elaborar una tabla de amortización  con cuota lineal creciente de   12.000COP para la suma de   100.000COP en 4 pagos anuales y una tasa del 8\% nominal anual mes vencido.\\
b. Elaborar una tabla amortización una cuota lineal decreciente de   12.000COP.\\

\begin{itemize}
	\item A. Crecimiento lineal de la cuota de   12.000COP
	\item B. Decremento lineal de la cuota de  12.000COP \\
\end{itemize}

\textbf{Solución}\\
	%La tabla ira centrada
	\begin{center}
		\renewcommand{\arraystretch}{1.4}% Margenes de las celdas
		\begin{longtable}[H]{|c|c|c|}
			\hline
			%Definimos el color de la primera fila
			\rowcolor[HTML]{FFB183}
			%%%%% INICIO ASIGNACIÓN FECHA FOCAL %%%%%%%
			%%%%%%%%%% INICIO TITULO
			\multicolumn{3}{|c|}{\cellcolor[HTML]{FFB183}\textbf{1. Asignación período focal}}  \\ \hline
			\multicolumn{3}{|c|}{$pf=4 \textit{naav}$} \\ \hline
			%%%%%%%%%% FIN TITULO
			%%%%% INICIO DECLARACIÓN DE VARIABLES %%%%%%%
			%%%%%%%%%% INICIO TITULO
			\multicolumn{3}{|c|}{\cellcolor[HTML]{FFB183}\textbf{2. Declaración de variables}}   \\ \hline
			%%%%%%%%%% FIN TITULO
			%%%%%%%%%% INICIO DE MATEMÁTICAS
			%Cada & hace referencia al paso de la siguiente columna
			$\hspace{1.5cm}VP=  100{.}000COP\hspace{1.5cm}$ & $\hspace{1.5cm}i=8\%\textit{naav}\hspace{1.5cm}$ & $R= ?COP $ \\
			$L=  12{.}000COP$ & $n=4\textit{naav}$ & $$ \\\hline
			
			%%%%%%%%%% FIN DE MATEMÁTICAS
			%%%%% FIN DECLARACIÓN DE VARIABLES
			
			
			%%%%% INICIO FLUJO DE CAJA
			\rowcolor[HTML]{FFB183}
			\multicolumn{3}{|c|}{\cellcolor[HTML]{FFB183}\textbf{3. Diagrama de flujo de caja}} \\ \hline
			%Mezclamos 3 columnas y pondremos el dibujo
			%%%%%%%%%%%%% INSERCIÓN DE LA IMAGEN
			\multicolumn{3}{|c|}{ \includegraphics[trim=-5 -5 -5 -5 , scale=0.6]{6_Capitulo/ejemplos/5/capitulo6Ejemplo5a.pdf} }
			
			\\ \hline
			%%%%%%%%%%%%% FIN INSERCIÓN DE IMAGEN
			%%%%%FIN FLUJO DE CAJA
			
			%%%%% INICIO DECLARACIÓN FORMULAS
			%%%%%%%%%%% INICIO TITULO
			\rowcolor[HTML]{FFB183}
			\multicolumn{3}{|c|}{\cellcolor[HTML]{FFB183}\textbf{4. Declaración de fórmulas}}    \\ \hline
			%%%%%%%%%%% FIN TITULO
			%%%%%%%%%%% INICIO MATEMÁTICAS
			
			\multicolumn{3}{|c|}{$VP=R(\frac{1-(1+i)^{-n}}{i})+\frac{L}{i}[\frac{1-(1+i)^{-n}}{i}-n(1+i)^{-n}] \hspace{0.4 cm} \textit{Valor presente gradiente aritmético}$} \\
			\multicolumn{3}{|c|}{$R_n=R_1+(n-1)L \hspace{0.4 cm} \textit{Valor flujo de un gradiente aritmético}$} \\ \hline
			
			%%%%%%%%%% FIN MATEMÁTICAS
			%%%%%% INICIO DESARROLLO MATEMÁTICO
			\rowcolor[HTML]{FFB183}
			%%%%%%%%%%INICIO TITULO
			\multicolumn{3}{|c|}{\cellcolor[HTML]{FFB183}\textbf{5. Desarrollo matemático}}       \\ \hline
			%%%%%%%%%% FIN TITULO
			%%%%%%%%%% INICIO MATEMÁTICAS
			\multicolumn{3}{|c|}{$  100{.}000COP=R(\frac{1-(1+0.08)^{-4}}{0.08})+[\frac{  12{.}000COP}{0.08}(\frac{1-(1+0.08)^{-4}}{0.08})-4(1+0.08)^{-4}]\hspace{0.4cm}\textit{Ecuación de equv.}$} \\
			\multicolumn{3}{|c|}{$R_1=  13{.}344COP$} \\
			\multicolumn{3}{|p{\textwidth}|}{Las demás cuotas se pueden calcular con la fórmula del último término del gradiente lineal o aritmético:} \\ 
			\multicolumn{3}{|l|}{$R_{n} = R_{1} + (n-1)L$} \\ \multicolumn{3}{|l|}{$R_{2} =  13{.}344,56 COP +   12{.}000 COP=   25{.}344,56COP$} \\ 
			\multicolumn{3}{|l|}{$R_{3} =  13{.}344COP +2 ( 12{.}000COP ) =  37{.}344,56COP $} \\ 
			\multicolumn{3}{|l|}{$R_{4} =   13{.}344COP+3 (  12{.}000COP) =   49{.}344COP$} \\ \hline
			\multicolumn{3}{|p{\textwidth}|}{Con los anteriores datos se puede elaborar la tabla de amortización, en la misma forma como se trabajó con las series uniformes.} \\ \hline
			
			
			%%%%%%%%%% FIN MATEMÁTICAS
			%%%%%% FIN DESARROLLO MATEMÁTICO
			%%%%%% INICIO RESPUESTA
			\rowcolor[HTML]{FFB183}
			%%%%%%%%%%INICIO TITULO
			
		\end{longtable}
	\end{center}
		La tabla de armortización es: 

	\begin{center}
	\begin{spacing}{1.1}
		\begin{tabular}{|p{1cm}|p{2cm}|p{2cm}|p{2cm}|p{3cm}|}
			\hline
			\rowcolor{white!50}
			\textbf{n\ (1)} & \textbf{Saldo Deuda (2)=(2)-(5) (COP)} & \textbf{Intereses  (3)=(2)(i) (COP)} & \textbf{Pago\ (4)=  R (COP)} & \textbf{Amortización  (5)=(4)-(3)(COP)} \\ \hline
			%preguntar porque valores de pdf no son iguales a la tabla
			
			0               &   100{.}000,00                     & ---------                      & ---------               & ---------                          \\ \hline
			1               &   94{.}655,44                      &   8{.}000,00                     &   13{.}344,56             &   5{.}344,56                         \\ \hline
			2               &   76{.}883,31                      &   7{.}572,43                     &   25{.}344,56             &   17{.}772,13                        \\ \hline
			3               &   45{.}689,41                      &   6{.}150,66                     &   37{.}344,56             &   31{.}193,90                        \\ \hline
			4               &   0,00                           &  3{.}655,15                     &   49{.}344            &   45{.}689,41                        \\ \hline
		\end{tabular}
	\end{spacing}
	\end{center}
	%%%%%%%%%%%%%%%%%%%%%%%%%%FIN EJERCICIO 5 %%%%%%%%%%%%%%%%%%%%%%%%%%%

\begin{flushleft}
	\textbf{Solución literal b.}\\
\end{flushleft}

%La tabla ira centrada
\begin{center}
	\renewcommand{\arraystretch}{1.4}% Margenes de las celdas
	\begin{longtable}[H]{|c|c|c|}
		\hline
		%Definimos el color de la primera fila
		\rowcolor[HTML]{FFB183}
		%%%%% INICIO ASIGNACIÓN FECHA FOCAL %%%%%%%
		%%%%%%%%%% INICIO TITULO
		\multicolumn{3}{|c|}{\cellcolor[HTML]{FFB183}\textbf{1. Asignación período focal}}  \\ \hline
		\multicolumn{3}{|c|}{$pf=0 \textit{ pav}$} \\ \hline
		%%%%%%%%%% FIN TITULO
		%%%%% INICIO DECLARACIÓN DE VARIABLES %%%%%%%
		%%%%%%%%%% INICIO TITULO
		\multicolumn{3}{|c|}{\cellcolor[HTML]{FFB183}\textbf{2. Declaración de variables}}   \\ \hline
		%%%%%%%%%% FIN TITULO
		%%%%%%%%%% INICIO DE MATEMÁTICAS
		%Cada & hace referencia al paso de la siguiente columna
		$\hspace{1.5cm}VP=  100{.}000COP\hspace{1.5cm}$ & $\hspace{1.5cm}i=8\%\textit{naav}\hspace{1.5cm}$ & $R= ?COP $ \\
		$L=  -12{.}000COP$ & $n=4\textit{ pav}$ & $$ \\\hline
		
		%%%%%%%%%% FIN DE MATEMÁTICAS
		%%%%% FIN DECLARACIÓN DE VARIABLES
		
		
		%%%%% INICIO FLUJO DE CAJA
		\rowcolor[HTML]{FFB183}
		\multicolumn{3}{|c|}{\cellcolor[HTML]{FFB183}\textbf{3. Diagrama de flujo de caja}} \\ \hline
		%Mezclamos 3 columnas y pondremos el dibujo
		%%%%%%%%%%%%% INSERCIÓN DE LA IMAGEN
		\multicolumn{3}{|c|}{ \includegraphics[trim=-5 -5 -5 -5 , scale=0.6]{6_Capitulo/ejemplos/5/capitulo6Ejemplo5b.pdf} }
		
		\\ \hline
		%%%%%%%%%%%%% FIN INSERCIÓN DE IMAGEN
		%%%%%FIN FLUJO DE CAJA
		
		%%%%% INICIO DECLARACIÓN FORMULAS
		%%%%%%%%%%% INICIO TITULO
		\rowcolor[HTML]{FFB183}
		\multicolumn{3}{|c|}{\cellcolor[HTML]{FFB183}\textbf{4. Declaración de fórmulas}}    \\ \hline
		%%%%%%%%%%% FIN TITULO
		%%%%%%%%%%% INICIO MATEMÁTICAS
		
		\multicolumn{3}{|c|}{$VP=R(\frac{1-(1+i)^{-n}}{i})+\frac{L}{i}[\frac{1-(1+i)^{-n}}{i}-n(1+i)^{-n}] \hspace{0.4 cm} \textit{Valor presente gradiente aritmético}$} \\
		\multicolumn{3}{|c|}{$R_n=R_1+(n-1)L \hspace{0.4 cm} \textit{Valor flujo de un gradiente aritmético}$} \\ \hline
		
		%%%%%%%%%% FIN MATEMÁTICAS
		%%%%%% INICIO DESARROLLO MATEMÁTICO
		\rowcolor[HTML]{FFB183}
		%%%%%%%%%%INICIO TITULO
		\multicolumn{3}{|c|}{\cellcolor[HTML]{FFB183}\textbf{5. Desarrollo matemático}}       \\ \hline
		%%%%%%%%%% FIN TITULO
		%%%%%%%%%% INICIO MATEMÁTICAS
		\multicolumn{3}{|c|}{$  100{.}000COP=R(\frac{1-(1+0.08)^{-4}}{0.08})+[\frac{  -12{.}000COP}{0.08}(\frac{1-(1+0.08)^{-4}}{0.08})-4(1+0.08)^{-4}]\hspace{0.4cm}\textit{Ecuación de equv.}$} \\ \hline
		
		%%%%%%%%%% FIN MATEMÁTICAS
		%%%%%% FIN DESARROLLO MATEMÁTICO
		%%%%%% INICIO RESPUESTA
		\rowcolor[HTML]{FFB183}
		%%%%%%%%%%INICIO TITULO
		\multicolumn{3}{|c|}{\cellcolor[HTML]{FFB183}\textbf{6. Respuesta}}   \\ \hline
		%%%%%%%%%% FIN TITULO
		%%%%%%%%%% INICIO RESPUESTA MATEMÁTICA
		\multicolumn{3}{|c|}{El valor es $R_1=  47{.}039COP$} \\ \hline
		%%%%%%%%%% FIN MATEMÁTICAS
		%%%%%% FIN RESPUESTA
	\end{longtable}
\end{center}
	La tabla de armortización es: 
	\begin{center}
	\begin{spacing}{1.1}
		\begin{tabular}{|p{1cm}|p{2cm}|p{2cm}|p{2cm}|p{3cm}|}
			\hline
			\rowcolor{white!50}
			\textbf{n\ (1)} & \textbf{Saldo Deuda (2)=(2)-(5) (COP)} & \textbf{Intereses  (3)=(2)(i) (COP)} & \textbf{Pago\ (4)= R (COP)  } & \textbf{Amortización  (5)=(4)-(3) (COP)} \\ \hline
			
			0               &   100{.}000,00                     & -                     & -               & -                         \\ \hline
			1               &   60{.}960,40                      &    8{.}000,00                    &  47{.}039,60             &   39{.}039,60                        \\ \hline
			2               &   30{.}797,63                      &    4{.}876,83                    &  35{.}039,60             &   30{.}162,77                        \\ \hline
			3               &   10{.}221,84                      &   2{.}463,81                     &   23.039,60             &   20{.}575,79                        \\ \hline
			4               &   0,00                           &   817,76                       &   11{.}039,60             &   10{.}221,84                        \\ \hline
		\end{tabular}
	\end{spacing}
\end{center}
%%%%%%%%%%%%%%%%%%%%%%%%%%FIN EJERCICIO 5 %%%%%%%%%%%%%%%%%%%%%%%%%%%
