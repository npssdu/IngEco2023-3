%----------------------------------------------------------------------------------------
%	Ejercicios
%----------------------------------------------------------------------------------------
\chapter*{Capítulo 4}
\addcontentsline{toc}{chapter}{\textcolor{ocre}{Capítulo 4}}


\begin{itemize}

 \item 1. Hallar el monto y el valor presente de 20 pagos de 200.000 COP c/u, suponga una tasa del 18\% "nominal anual año vencido".\\
       \medskip

 \item 2. Para la compra de un automóvil que vale 6.000.000 COP; se exige una cuota inicial del 40\% y el resto se cancela en 36 cuotas mensuales, ¿a cuánto ascenderá la cuota, si los intereses son del 3.5\% periodo mes vencido\\
       \textbf{Respuesta:} 177.423 COP\\
       \medskip

 \item 3. Si en el problema anterior se ofrecen 2 cuotas extraordinarias: la primera de
       350.000 COP en el mes 5, y la segunda de 500.000 COP, en el mes 18, ¿cuál será el valor de la cuota ordinaria?\\
       \textbf{Respuesta:} 149.633 COP\\
       \medskip

 \item 4. Una persona va a comprar una máquina que vale 800.000 COP, con el objeto de poder disponer de esa cantidad el 15 de diciembre de 1989. Comienza a hacer depósitos mensuales de "R" COP, en un fondo que paga el 30\% nominal anual mes vencido. Si el primer depósito lo hace el 15 de febrero de 1988, Hallar el valor del depósito mensual.\\
       \textbf{Respuesta:} 26.157 COP\\
       \medskip

 \item 5. Un documento estipula pagos trimestrales de 100.000 COP iniciando el primer pago el 20 de enero de 1987 y terminando el 20 de julio de 1995: Si se desea cambiar este documento por otro que estipule pagos trimestrales de "R" COP, comenzando el 20 de abril de 1988 y terminando el 20 de julio de 1989, Hallar el valor de la cuota. Suponga una tasa del 20\% nominal anual trimestre vencido. \textbf{Sugerencia}: El valor de los documentos debe ser igual en el punto que escoja como período focal.\\
       \medskip

 \item 6. Una persona se compromete a pagar 600.000 COP mensuales, a partir del 8 de julio de 1988 hasta el 8 de diciembre de 1989. Se propone hacer depósitos mensuales de "R" COP c/u, en una cuenta de ahorros que como mínimo le garantiza el 1.5\% periódo mes vencido. Si el primer depósito lo efectúa el 8 de marzo de 1986, ¿cuál será el valor de "R" COP (valor de la serie uniforme), suponiendo que el último depósito lo hará:\\
       a, El 8 de diciembre de 1989\\
       b. El 8 de julio de 1988\\
       c. El 8 de junio de 1988\\
       d. El 8 de abril de 1987\\
       \medskip

 \item 7. Una deuda de 8'000.000 COP va a ser cancelado en pagos trimestrales de 780.000 COP durante tanto tiempo como fuere necesario. Suponiendo una tasa del 30\% nominal anual trimestre vencido.\\
       a. ¿Cuántos pagos de 780.000 COP deben hacerse?\\
       b. ¿Con qué pago final hecho 3 meses después del último pago de 780.000 COP cancelará la deuda?\\
       \medskip

 \item 8. Resuelva el problema anterior si la tasa es del 42\% nominal anual trimestre vencido. Justifique su respuesta desde el punto de vista matemático y desde el punto de vista financiero.\\
       \medskip

 \item 9. Desean reunirse exactamente 600.000 COP mediante depósitos mensuales de 10.000 COP, en un fondo que paga el 36\% nominal anual mes vencido.\\
       a. ¿Cuántos depósitos de 10.000 COP deberán hacerse?\\
       b. ¿Qué depósito adicional hecho conjuntamente con el último depósito de 10.000 COP completará los 600.000 COP?\\
       c. ¿Qué depósito adicional hecho un mes después del último depósito de 10.000 COP completará los 600.000 COP?\\
       \medskip
 \item 10. Resolver el problema anterior incluyendo un depósito adicional de 70.000 COP en el período 10.\\
       \medskip

 \item 11. Para cancelar una deuda de 8'000.000 COP, con intereses al 24\% nominal anual mes vencido se hacen pagos mensuales de 300.000 COP cada uno.\\
       a.¿Cuántos pagos de 300.000 COP deben hacerse?\\
       b.¿Con qué pago adicional hecho conjuntamente con el último pago de 300.000 COP se cancelará la deuda?\\
       c. ¿Qué pago adicional hecho un mes después del último pago de 300.000 COP cancelará la deuda?\\
       \medskip

 \item 12. Resolver el problema anterior suponiendo que se hace un pago adicional de 10.000 COP con la décima cuota.\\
       \medskip

 \item 13. Una máquina cuesta al contado 600.000 COP, para promover las ventas, se ofrece que puede ser vendida en 24 cuotas mensuales iguales, efectuándose la primera el día de la venta. Si se carga un interés del 3\% período mes vencido . Calcular el valor de cada pago.\\
       \textbf{Respuesta:} 34.397 COP\\
       \medskip

 \item 14. Un fondo para empleados presta a un socio la suma de 2´000.000 COP para ser pagado en 3 años, mediante cuotas mensuales uniformes, con intereses sobre saldos al 24\% nominal anual mes vencido. Si en el momento de pagar la sexta cuota, decide pagar en forma anticipada las cuotas 7, 8 y 9:\\
       a. ¿cuál debe ser el valor a cancelar al vencimiento de la sexta cuota?\\
       b. ¿cuál debe ser el valor de los intereses descontados?\\
       \textbf{Respuestas:} a. 304.752 COP \hspace{1,5cm}b. 9.111 COP\\
       \medskip

 \item 15. Una  persona adopta un plan de ahorros del fondo ABC, que establece depósitos mensuales de 100.000 COP, comenzando el primero de febrero de 1986 hasta el primero de abril de 1987 y, depósitos mensuales de 200.000 COP, desde el primero de mayo de 1987 hasta el primero de diciembre de 1987. El capital así reunido permanecerá en el fondo hasta el primero de junio de 1988, fecha en la cual le será entregado al suscriptor junto con intereses calculados al 12\% nominal anual mes vencido. \\
       Por razones comerciales la junta directiva del fondo ABC decidió que, a partir del primero de octubre de 1986, el fondo pagará a todos sus clientes de ahorros el 18\% nominal anual mes vencido. ¿Cuál será el capital que, el primero de junio de 1988, le entregarán a la persona que ha decidido adoptar el plan?\\
       \medskip

 \item 16. Un contrato de arriendo por un año establece el pago de 200.000 COP mensuales al principio de cada mes. Si ofrecen cancelar todo el contrato a su inicio, ¿cuánto deberá pagar?. Suponiendo:\\
       a. tasa del 30\% Nominal Anual mes anticipado.\\
       b. tasa 3\% período mes anticipado \\
       \medskip

 \item 17. Una máquina produce 2.000 unidades mensuales las cuales deben venderse a 800 COP c/u. El estado actual de la máquina es regular y si no se repara podría servir durante 6 meses más y luego desecharla, pero si hoy le hacemos una reparación total a un costo de 8´000.000 COP, se garantizaría que la máquina podría servir durante un año contado a partir de su reparación. Suponiendo una tasa del 4\% período mes anticipado ¿será aconsejable repararla?\\
       \textbf{Respuesta:} No es aconsejable repararla\\
       \medskip

 \item 18. Elaborar una tabla para amortizar la suma de 3´000.000 COP en pagos trimestrales durante 15 meses con una tasa del 46\% nominal anual trimestre vencido.\\
       \textbf{Respuesta parcial:} Cuota trimestral 821.945 COP \\
       \medskip

 \item 19. Elaborar una tabla para capitalizar la suma de 2´000.000 COP mediante depósitos semestrales durante 3 años. Suponga una tasa del 42\% nominal anual semestre vencido\\
       \textbf{Respuesta parcial:} Depósito semestral 196.406 COP \\
       \medskip

 \item 20. Una persona desea reunir 800.000 COP mediante depósitos mensuales de "R" COP c/u durante 5 años en una cuenta que paga el 30\% nominal anual mes vencido. ¿Cuál es el total de intereses ganados hasta el mes 30?\\
       \textbf{Respuesta:} 81.786 COP\\
       \medskip

 \item 21. Para cancelar una deuda de 2´000.000 COP con intereses al 36\% nominal anual mes vencido se hacen pagos mensuales de  "R" COP c/u, durante 15 años.\\
       a. Calcular el valor de la deuda después de haber hecho el pago número 110\\
       b. Calcular el total de los intereses pagados hasta el mes 110\\
       \textbf{Sugerencia:}para la parte a. calcule el valor presente en el mes 110  de los 70 pagos que falta por cancelar, para la parte b. halle la diferencia entre el total pagado y el total amortizado.
       \textbf{Respuestas:} a.  1'755.992 COP \hspace{1,5 cm} b. 6'388.424 COP\\
       \medskip

 \item 22. Se necesita 1´000.000 COP, para realizar un proyecto de ampliación de una bodega, una compañía A ofrece prestar el dinero, pero exige que le sea pagado en 60 cuotas mensuales vencidas de 36.133 COP c/u. La compañía B ofrece prestar el dinero, pero para que le sea pagado en 60 pagos mensuales de 19.000 COP c/u y dos cuotas adicionales así: la primera de 250.000 COP, pagadera al final del mes 12, la segunda, de 350.000 COP, pagadera al final del mes 24. Hallar la tasa periódica mensual vencida que cobra cada uno, para decidir que préstamo debe utilizar.\\
       \textbf{Respuesta:} A. 3\% período mes vencido  \hspace{0,5 cm}              B. 2,35\% período mes vencido\\
       Utilice la compañía B.\\
       \medskip

 \item 23. Un equipo de sonido cuesta 4'000.000 COP al contado, pero puede ser cancelado en 24 cuotas mensuales de 330.000 COP c/u efectuándose la primera el día de la venta. ¿Qué tasa período mes anticipado se está cobrando?\\
       \medskip

 \item 24. ¿A qué tasa nominal, nominal anual mes vencido , está siendo amortizada una deuda de 3'000.000 COP, mediante pagos mensuales de 100.000 COP, durante 4 años?\\
       \medskip

 \item 25. ¿A qué tasa nominal anual trimestre vencido , está reuniéndose un capital de 4'000.000 COP mediante depósitos trimestrales de 200.000 COP c/u durante 3 años?\\
       \medskip

 \item 26. Una entidad financiera me propone que le deposite mensualmente 100.000 COP durante 3 años comenzando el primer depósito el día de hoy y me promete devolver al final de este tiempo la suma de 7'000.000 COP. ¿Qué tasa período mes anticipado me va a pagar?\\
       \medskip

 \item 27. Un señor compró un automóvil, dando una cuota inicial del 20\% y el saldo lo cancela con cuotas mensuales de 317.689 COP durante 3 años. Después de efectuar el pago de la cuota 24 ofrece cancelar el saldo de la deuda de un solo contado y le dicen que su saldo en ese momento asciende a la suma de 3'060.929 COP.\\
       a. Calcular la tasa (periódica mes vencido) que le están cobrando.\\
       b. Calcular la tasa "nominal anual año vencido" equivalente que le cobran.\\
       c. ¿Cuál es el costo total del automóvil?\\
       \textbf{Respuestas:} a. 3.55\% período mes vencido  \hspace{0,5cm} b. 52\% "nominal anual año vencido"  \hspace{0,5cm} c.8´000.000 COP

\end{itemize}\
