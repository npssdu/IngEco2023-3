\chapter*{Capítulo 3}
\addcontentsline{toc}{chapter}{\textcolor{ocre}{Capítulo 3}}


\begin{itemize}

 \item 1. Se  constituye  un  CDT  a  180  días  por \$650.000 COP,  con  una  tasa  del  26\% natv (nominal anual trimestre vencido) y teniendo  en  cuenta  que  la retención en la fuente es del 7\% naav (nominal anual año vencido), determinar:\\

       a. La tasa de interés(rentabilidad) antes de impuestos.\\
       b. La tasa de interés (rentabilidad) después de impuestos\\
       c. El valor en pesos que le entregan al vencimiento.\\
       d. Suponiendo una inflación del 18\% anual efectiva, determinar la tasa real obtenida.\\
       \textbf{Respuesta:} a. i = 28,647\% pav (periodo año vencido); b. i = 26,6524\% pav (periodo año vencido); c. F_{Neto} = \$731.139,02 COP; d. i = 7.224\% pav (periodo año vencido)\\
       \medskip

 \item 2. Un inversionista desea  obtener  una  rentabilidad  real  del  8\%  naav (nominal anual año vencido) ¿A qué tasa periódica debe invertir suponiendo que la inflación va a ser del 18\% naav (nominal anual año vencido)?

       \textbf{Respuesta:} 27,44\% naav (nominal anual año vencido)\\
       \medskip

 \item 3. Un  artículo  es  fabricado  en  Estados  Unidos  y  se  vende  en  Colombia  en \$500.000 COP ¿Cuánto  valdrá  el  artículo  en  Colombia  y  en  Estados  Unidos  al final  de  un  año,  suponiendo  los  siguientes  índices  económicos: cambio actual \$1 USD =  \$2.000 COP, inflación en Estados Unidos 3\% naav (nominal anual año vencido), devaluación del peso 18\% naav (nominal anual año vencido)?\\
       \textbf{Respuesta:} F_{C} = \$60.770 COP y F_{EU} = \$25,75 US\\
       \medskip

 \item 4. Un  artículo  es  fabricado en  Colombia  y  cuesta \$680.000 COP,  cuando  el cambio es de \$1 USD = \$2.000 COP. Suponiendo que el IPP de este sector en Colombia es del 22\% naav (nominal anual año vencido), y que la devaluación del peso frente al dólar sea del 18\% naav (nominal anual año vencido), hallar el precio del mismo artículo en cada país al final de un año.\\
       \textbf{Respuesta:} F_{C} = \$82.960 COP y F_{EU} = \$35,15 US\\
       \medskip

 \item 5. Dos  inversionistas  de  origen  alemán,  uno  residente  en  Alemania  y el  otro residente  en  Colombia,  han  decidido  realizar  un  negocio  en  Alemania  y cada uno aportará el 50\%. El negocio exige una inversión inicial de marcos \$300.000  DMCOP   y   al   final   de   3   años   devolverá   la   suma   de   marcos \$400.000 DMCOP. Hallar las tasas totales y reales para cada uno de los socios suponiendo  que  los  siguientes  indicadores  económicos  se  mantuvieron estables durante los 3 años.\\

       a.tasa promedio de inflación en Colombia 22\% naav (nominal anual año vencido)\\
       b.tasa promedio de inflación en Alemania 2\% naav (nominal anual año vencido)\\
       c.   tasa  de  devaluación  del  peso  frente  al  dólar:  primer  año  18\% naav (nominal anual año vencido), segundo  año  20\% naav (nominal anual año vencido)  y  tercer  año  17\% naav (nominal anual año vencido),  devaluación  marco frente  al  dólar:  años  1  y  2  el  2\% naav (nominal anual año vencido),  para  el  tercer  año  hay  una revaluación del 3\% naav (nominal anual año vencido)\\
       d. cambio actual \$1 USCOP  =  \$2,23 COP; \$1 USCOP = \$1.300 COP\\
       \medskip
       \textbf{Respuestas:} En marcos 10.06\% naav (nominal anual año vencido) y 7.9\% naav (nominal anual año vencido); en pesos: 29,85\% naav (nominal anual año vencido) y 6,43\% naav (nominal anual año vencido).\\
       \medskip

 \item 6. El señor Yukimoto residente en el Japón y Mr.Jones residente en Estados Unidos  se  asocian para comprar un banco en Colombia, El valor de cada acción del banco es de \$900.000 COP pesos/acción y esperan venderla al final de 3 meses en \$900.700 COP pesos/acción. (Trabajar con 5 decimales).\\

       a. Cálcule la tasa de interés anual efectiva y la rentabilidad real(tasa de interés real) anual de cada uno de los socios\\
       b. ¿Cuánto tendrá cada uno en su respectiva moneda al final de los 3 meses?. Tome en cuenta la siguiente información:\\

       Inflación en: Colombia 18\% naav (nominal anual año vencido), en Estados Unidos 3.5\% naav (nominal anual año vencido), en Japón 2.3\%  naav (nominal anual año vencido) tasa de devaluación del peso frente al dólar 22\%  naav (nominal anual año vencido) tasa de devaluación del dólar frente al Yen 1\% naav (nominal anual año vencido) Cambio actual \$1 USCOP = \$2.000 COP; \$1 USCOP = \$105 YEN\\
       \textbf{Respuesta:} a. Mr. Jones: i = 10,58\% naav (nominal anual año vencido) y i_{R} = 6,86\% naav (nominal anual año vencido); Mr. Yukimoto: i = 9,49\% naav (nominal anual año vencido) y i_{R} = 0,03\% naav (nominal anual año vencido)\\
       b. F_{yen} = \$241,626 YEN y F_{USD} = \$2,307 USD\\
      \medskip

 \item 7. Si en el problema anterior el valor del banco es de ochenta mil millones de pesos y Yukimoto  participa en el 40\% de la compra y Mr. Jones participa con el resto, determinar la cantidad que  recibirá c/u en su respectiva moneda.\\
      \textbf{Respuesta:} Mr. Jones: F_{J} = \$1.718.519,191 YEN; Mr. Yukimoto: F_{J} = \$24.612,15 US\\
      \medskip

 \item 8. En el país A cuya moneda es el ABC, un par de zapatos vale \$240.000 de ABC, existe una inflación  del 22\% naav (nominal anual año vencido) y el cambio actual es de \$1 USCOP = ABC \$1.000 ABC. En el país X rige el dólar americano y se prevé  una inflación promedio del 6.5\% naav (nominal anual año vencido). Al final de un año ¿cuál debe ser la tasa de devaluación en A con respecto al dólar a fin de no perder competitividad en los mercados de X?\\
      \textbf{Respuesta:} Tasa devaluativa = 14,55\% pav (periodo año vencido)\\
       \medskip

 \item 9. Un inversionista desea que todas sus inversiones le den una rentabilidad real del 5\% naav (nominal anual año vencido) ¿Qué  tasa anual efectiva debe ofrecérsela si la inflación esperada es del 17\% naav (nominal anual año vencido) de forma tal que satisfagan  los deseos del inversionista?
       \textbf{Respuesta:} 26.36\% naav (nominal anual año vencido)\\
       \medskip

 \item 10. Un ahorrador consigna en una corporación de ahorro y vivienda la suma de \$300.000 COP el día 1 de  marzo y el día 20 de junio consigna COP \$200.000.  ¿Cuánto  podrá  retirar  el  31  de  agosto  si la corporación  paga el  27\% naav (nominal anual año vencido) de corrección monetaria para los meses de marzo y abril y el 25\% naav (nominal anual año vencido)  para el resto del período (mayo, junio, julio y agosto).\\
       a. Elabore los cálculos en pesos\\
       b. Elabore los cálculos en UPAC sabiendo que el primero de marzo upac \$1 upac = \$6 650 COP\\
       \textbf{Respuestas:} \$545 389 COP naav (nominal anual año vencido)\hspace{0,5cm} \$73.1415 UPAC\\
       \medskip

 \item 11. Se estima que la corrección monetaria del primer año será del 18\% "nominal anual año vencido" y la del segundo año del 17\% naav (nominal anual año vencido):\\
       a.Calcular la cantidad que antes de impuestos le entregarán a un inversionista que invierte la suma de COP 800.000 a dos años en una cuenta de ahorros en UPAC que le garantiza pagar la corrección monetaria más el 4\% naav (nominal anual año vencido) de interés sobre los UPAC.\\
       b. Calcule la rentabilidad (tasa de interés "nominal anual año vencido") obtenida antes de impúestos que el cambio actual es \$1 UPAC = \$14.000 COP\\
       c. Si la retención en la fuente es del 7\% sobre los intereses, calcular la rentabilidad (tasa de interes naav (nominal anual año vencido)) después de los impuestos\\
       d. Calcular la cantidad final que le entregarán después de impuestos\\
       \textbf{Respuestas:}a. \$1 194 605.57 COP \hspace{0,5cm} b. 22.199\% naav (nominal anual año vencido)\hspace{0,5cm} c. 21,876\% naav (nominal anual año vencido)\hspace{0,5cm} d. \$1 188 296.78 COP\\
       \medskip

 \item 12. Hallar la tasa anual efectiva de;\\
       a. DTF +6 puntos\\
       b. IPC +7 puntos\\
       c Libor +8 puntos\\
       Asuma que: DTF = 15\% nata (nominal anual trimestre vencido), IPC = 10\% nata, Libor = 5.14\% nasa (nominal anual semestre vencido)\\
       \textbf{Respuestas:} a.24.07\% naav (nominal anual año vencido)\hspace{0,5cm} b.17.7\% naav (nominal anual año vencido)\hspace{0,5cm} c.13.57\% naav (nominal anual año vencido)\\
       \medskip

 \item 13. Suponiendo IPC = 8.5\% naav (nominal anual año vencido), CM= 12\% (CM= corrección monetaria), DTF = 15\% nata, TCC = 15.5\% nata, TBS (CF 180 días) = 19.27\% A.E., TBS(Bancos 360 días) = 19.19\% naav (nominal anual año vencido) Hallar X de las siguientes igualdades:\\
       \textbf{Observación:} TBS (CF 180 días) significa tasa básica del sector corporaciones financieras a 180 días.\\

       a. IPC+10 = CM+x\\
       b. CM+14 = TCC+X\\
       c. DTF +8.6 = IPC+X\\
       d. TBS(CF 180 días) + 6 = DTF+x\\
       e. TCC+3.5 = DTF+X\\
       f. IPC+4 = DTF+X\\
       \textbf{Respuestas:}a.6.56\% naav (nominal anual año vencido)\hspace{0,5cm} b.8.2\% nata (nominal anual trimestre vencido)\hspace{0,5cm} c.17.55\%A.E
       \\d.7.775\% nata (nominal anual trimestre vencido)\hspace{0,5cm} e.4\% nata (nominal anual trimestre vencido)\hspace{0,5cm} f. -3.1\% nata (nominal anual trimestre vencido)\\
       \medskip

 \item 14. Asumiendo que $i_{dev}$  = 25\%, IPC = 9\% naav (nominal anual año vencido), Prime Rate = 8.25\% naav (nominal anual año vencido), DTF = 14.5\% nata (nominal anual trimestre vencido), Libor = 5\% naav (nominal anual año vencido), resolver las siguientes ecuaciones:\\
       $i_{DEV}$  + 10 = IPC +X\\
       $i_{DEV}$  +(Prime+ 200 p.b.) = DTF +X\\
       $i_{DEV}$  + (Libor + 500 p.b.) = DTF +X\\
       \textbf{Respuestas:} a. 26.148\% naav (nominal anual año vencido)\hspace{0,5cm} naav (nominal anual año vencido). b. 16,32\% nata (nominal anual trimestre vencido)\hspace{0,5cm} c, 16,11\% nata (nominal anual trimestre vencido)\\
       \medskip

 \item 15. ¿Cuál es la rentabilidad efectiva anual del comprador (tasa de interés "nominal anual año vencido") y el precio de compra para el que adquiere una aceptación financiera a 180 días si se conserva hasta su maduración, se registra en bolsa a un precio de 86.225\% y la comisión de compra es del 0.5\% "nominal anual año vencido" en rentabilidad?\\
       \textbf{Respuestas:} $i_{c} = 34\% naav (nominal anual año vencido)\hspace{0,5cm}  P_{c} = 86,37\%$\\
       \medskip

 \item 16. ¿Cuál es la comisión en pesos para el problema anterior suponiendo que la aceptación financiera tiene un valor nominal de \$278.000 COP?\\
       \textbf{Respuesta:} \$450 COP\\
       \medskip

 \item 17. ¿Cuál es la rentabilidad efectiva anual que obtiene un inversionista que adquiere en el mercado secundario una aceptación bancaria emitida a 90 días con un precio de registro de 97.254\% y le faltan 28 días para su maduración? Suponga una comisión de compra del 0.4\% naav (nominal anual año vencido) en rentabilidad. base 360.\\
       \textbf{Respuesta:} 42,645\% naav (nominal anual año vencido)\\
       \medskip

 \item 18. Un exportador recibe una aceptación bancaria por sus mercancías la cual vence en 180 días, tiene una tasa de emisión del 28\% nasv (nominal anual semestre vencido). El mismo día en que le entregan la aceptación la ofrece en bolsa. Si las comisiones de compra y de venta son de 0,4\% naav (nominal anual año vencido) y 0.6\% naav (nominal anual año vencido) respectivamente, calcular:\\
       a.La tasa de registro\\
       b.La tasa del comprador\\
       c.La tasa del vendedor\\
       d.El precio de registro\\
       e.El precio de compra\\
       \textbf{Respuestas:} a. 29.36\% naav (nominal anual año vencido)\hspace{0,5cm} b. 28.96\% naav (nominal anual año vencido)\hspace{0,5cm}  c. 29.96\% naav (nominal anual año vencido)\hspace{0,5cm} \\
       d. 87.922\% \hspace{0,5cm} e. 88.059\%.\\
       \medskip

 \item 19. Un inversionista compró el 14 de junio 98 una Aceptación Bancaria al 29.4\% naav (nominal anual año vencido) con vencimiento el 15 de mayo/99 por \$250 millones COP, un segundo inversionista está dispuesto a adquirirlo el día 10 de septiembre/98 a una tasa del 34\% naav (nominal anual año vencido).\\
       a.¿Cuál será la utilidad en pesos del primer inversionista?\\
       b.¿Cuál es la rentabilidad del primer inversionista? (use un interés comercial es decir un año de 360 días).\\
       \textbf{Respuestas:} a. \$7.598.455 COP\hspace{1.0cm} b. 17.14\% naav (nominal anual año vencido)\\
       \medskip

 \item 20. Resuelva el problema anterior pero el segundo inversionista lo adquiere al 23.5\% naav (nominal anual año vencido)\\
       \textbf{Respuestas:} a. \$19.296.120 COP\hspace{1,0cm} b. 47.8\% naav (nominal anual año vencido)\\
       \medskip

 \item 21. Suponga que el señor X posee una aceptación financiera con valor de vencimiento de \$6.758.000 COP y desea venderla en Bolsa faltándole 57 días para vencerse y quiere ganarse un 29.5\% y la adquiere el señor Y. Suponga que la comisión de venta y de compra son 0.5\% naav (nominal anual año vencido) y 0.47\% naav (nominal anual año vencido) respectivamente en rentabilidad. Base 365.\\
       a.¿Cuál es la tasa de registro?\\
       b.¿Cuál es el precio de registro?\\
       c. ¿Cuál la tasa que gana el señor Y?\\
       d.¿Cuál es el precio que paga el señor Y?\\
       e. ¿Cuál es la comisión de compra en pesos?\\
       \textbf{Respuestas:} a. $i_{R}= 29\%$ naav (nominal anual año vencido)\hspace{0,5cm} b. $P_{R}$ = \$6.494.534 COP\hspace{0,5cm}  c. $i_{c} = 28.53\%$ naav (nominal anual año vencido)\hspace{0,5cm}\\
       d. Pc= \$6.498.237 COP \hspace{0,5cm} e. \$3.703 COP\\
       \medskip

 \item 22. El señor XX posee una aceptación bancaria por valor de 10 millones COP y la vende en Bolsa faltándole 87 días para su maduración, la adquiere el señor YY y el cual desea ganar el 32\% después de comisión pero antes de impuestos. Si la comisión de compra es del 0.4\% naav (nominal anual año vencido) y la de venta el 0.375\% naav (nominal anual año vencido) usando un año de 360 días determinar:\\
       a.	La tasa de registro\\
       b.	El precio de registro\\
       c.	La tasa de cesión\\
       d.	El precio de cesión\\
       e.	El precio al comprador\\
       f.	El valor en pesos de la retención en la fuente\\
       g.	La cantidad que debe pagar YY\\
       h.	La cantidad que recibe XX\\
       i.	La rentabilidad después de impuestos que gana YY\\
       \textbf{Respuestas:} a. $i_{R}$ = 32.4\% naav (nominal anual año vencido)\hspace{0,5cm}  b. $P_{R}$ = \$9.344.234 COP\hspace{0,5cm}   c. 32.775\% naav (nominal anual año vencido)\hspace{0,5cm}  \\
       d. \$9.337.850 COP\hspace{0,5cm}   e.  $P_{c}$ = \$9 351 070 COP;  naav (nominal anual año vencido)\hspace{0,5cm}    f. \$45.904 COP;  \hspace{0,5cm}   g. \$9.396.974 COP; \hspace{0,5cm}\\
       h. \$9.383.754 COP \hspace{0,5cm} naav (nominal anual año vencido)\hspace{0,5cm}   i. 29.352\% naav (nominal anual año vencido).\\
       \medskip

 \item 23. En el problema 21 calcule el valor que recibe el vendedor y el valor que paga el comprador suponiendo que la retención en la fuente es del 7\% naav (nominal anual año vencido) sobre utilidades.\\
       \textbf{Respuestas:}\\
       El comprador paga \$6.516.680 COP,\\
       Vendedor recibe \$6.509.055 COP.\\
       \medskip

 \item 24. El 27 de abril de 1999 se compra una aceptación bancaria de 36 millones COP en el mercado bursátil, con vencimiento el 27 de julio de 1999 y con tasa de registro del 26\% naav (nominal anual año vencido). Si después de transcurridos 34 días la vende. ¿Qué precio se debe cobrar si el vendedor desea obtener una rentabilidad durante la tenencia del 26.5\% naav (nominal anual año vencido)?\\
       Base 365.\\
       \textbf{Respuesta}: $P_{v}$ = \$34.736.688 COP\\
       \medskip

 \item 25. Resuelva el problema anterior suponiendo que el corredor cobra una comisión del 0.1\% en rentabilidad y que de todas maneras el vendedor quiere ganarse el 26.6\% naav (nominal anual año vencido) durante la tenencia.\\
       \textbf{Respuesta:} $P_{v}$ = \$34.746.123 COP\\


\end{itemize}