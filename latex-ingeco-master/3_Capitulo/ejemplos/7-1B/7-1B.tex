\newpage
b) Para 40 días antes del vencimiento:

%La tabla ira centrada
\begin{center}
 \renewcommand{\arraystretch}{1.5}% Margenes de las celdas
 %Creación de la cuadricula de 3 columnas
 \begin{longtable}[H]{|p{0.5\linewidth}|p{0.5\linewidth}|}
  %Creamos una linea horizontal
  \hline
  %Definimos el color de la primera fila
  \rowcolor[HTML]{FFB183}
  %%%%% INICIO ASIGNACIÓN FECHA FOCAL %%%%%%%
  %%%%%%%%%% INICIO TITULO
  %Lo que se hace aquí es mezclar las 3 columnas en una sola
  \multicolumn{2}{|c|}{\cellcolor[HTML]{FFB183}\textbf{1. Asignación período focal}}                   \\ \hline
  %%%%%%%%%% FIN TITULO
  %%%%% INICIO DECLARACIÓN DE VARIABLES %%%%%%%
  \multicolumn{2}{|c|}{$pf = \frac{40}{365} \textit{ pdv}$ \newline}                                                      \\ \hline
  %%%%%%%%%% INICIO TITULO
  %Lo que se hace aquí es mezclar las 3 columnas en una sola
  \multicolumn{2}{|c|}{\cellcolor[HTML]{FFB183}\textbf{2. Declaración de variables}}                 \\ \hline
  %%%%%%%%%% FIN TITULO
  %%%%%%%%%% INICIO DE MATEMÁTICAS
  %Cada & hace referencia al paso de la siguiente columna
  $F =   100\ COP$                           & $V_{v} =\ ?\ COP  $                                               \\
  $i  = 30\%\textit{pdv}$              &                                                            \\
  $n = \frac{40}{365} = 0.109\% \ pdv  $ &    $P_{v} =  \ ?\ COP  $                                                                    \\ \hline
  %%%%%%%%%% FIN DE MATEMÁTICAS
  %%%%% FIN DECLARACIÓN DE VARIABLES


  %%%%% INICIO FLUJO DE CAJA
  \rowcolor[HTML]{FFB183}
  \multicolumn{2}{|c|}{\cellcolor[HTML]{FFB183}\textbf{3. Diagrama de flujo de caja}}                \\ \hline
  \multicolumn{2}{|c|}{ \includegraphics[trim=-78 -5 -78 -5]{3_Capitulo/img/ejemplos/7/capitulo3ejercicio7a1.pdf} }  \\ \hline


  %%%%%% FIN FLUJO DE CAJA
  \rowcolor[HTML]{FFB183}
  \multicolumn{2}{|c|}{\cellcolor[HTML]{FFB183}\textbf{4. Desarrollo matematico}}                    \\ \hline
  %Mezclamos 3 columnas y pondremos el dibujo
  %%%%%%%%%%%%% INSERCIÓN DE LA IMAGEN
  %Deberán descargar las imágenes respectivas del drive y pegarlas en la carpeta
  %n_capitulo/img/ejemplos/1/capitulo1ejemplo1.pdf  (el /1/ es el numero del ejemplo)
  \multicolumn{2}{|C{\linewidth}|}{

  $P_{v} =  100\ COP (1 + 0,3)^\frac{-40}{365} =  97,165\  COP $
 
  $V_v = ( 5{.}000{.}000)\ 0,97 = 4{.}850{.}000 \ COP $ \newline
  }                                                                                                  \\ \hline

  %%%%% INICIO DECLARACIÓN FORMULAS
  %%%%%%%%%%% INICIO TITULO
  \rowcolor[HTML]{FFB183}

  %%%%%% INICIO RESPUESTA
  \rowcolor[HTML]{FFB183}
  %%%%%%%%%%INICIO TITULO
  \multicolumn{2}{|c|}{\cellcolor[HTML]{FFB183}\textbf{5. Respuesta}}                                \\ \hline
  %%%%%%%%%% FIN TITULO
  %%%%%%%%%% INICIO RESPUESTA MATEMÁTICA
  \multicolumn{2}{|C{\textwidth}|}{\newline
 $V_v=  \ $4{.}850{.}000 \ COP\newline
  }                                                                                                  \\ \hline
  %%%%%%%%%% FIN MATEMÁTICAS
  %%%%%% FIN RESPUESTA
 \end{longtable}
 %Se crean dos lineas en blanco para que no quede el siguiente texto tan pegado
 %\newline \newline %USARLO SI CREES QUE ES NECESARIO
 
\end{center}
