
\textbf{Ejemplo 11}\\
Se hacen depósitos trimestrales con incremento del 5\% entre flujos, en una cuenta que paga el 5,25\% periódica trimestral vencida, con el fin de tener disponibles 500.000 COP el primero de enero de 1991, Si el primer depósito se hace el primero de abril de 1988 y el último el primero de julio de 1990, determinar el valor del primer depósito:\\

	%%%%%%%%%%%%%%%%%%% EJERCICIO 11 %%%%%%

%\newpage %USAR SOLO SI EL SOLUCIÓN QUEDA SOLO Y ES NECESARIO BAJARLO A LA SIGUIENTE PAGINA
\textbf{Solución.}\\
%La tabla ira centrada
\begin{center}
	\renewcommand{\arraystretch}{1.6}% Margenes de las celdas
	%Creación de la cuadricula de 3 columnas
	\begin{longtable}[H]{|c|c|c|}
		%Creamos una linea horizontal
		\hline
		%Definimos el color de la primera fila
		\rowcolor[HTML]{FFB183}
		%%%%% INICIO ASIGNACIÓN FECHA FOCAL %%%%%%%
		%%%%%%%%%% INICIO TITULO
		%Lo que se hace aquí es mezclar las 3 columnas en una sola
		\multicolumn{3}{|c|}{\cellcolor[HTML]{FFB183}\textbf{1. Asignación período focal}}  \\ \hline
		\multicolumn{3}{|c|}{$pf = \textit{12 ptv}$}   \\\hline
		%%%%%%%%%% FIN TITULO
		%%%%% INICIO DECLARACIÓN DE VARIABLES %%%%%%%
		%%%%%%%%%% INICIO TITULO
		%Lo que se hace aquí es mezclar las 3 columnas en una sola
		\multicolumn{3}{|c|}{\cellcolor[HTML]{FFB183}\textbf{2. Declaración de variables}}   \\ \hline
		%%%%%%%%%% FIN TITULO
		%%%%%%%%%% INICIO DE MATEMÁTICAS
		%Cada & hace referencia al paso de la siguiente columna
		\multicolumn{2}{|c|}{$\hspace{2 cm}VP=  500{.}000 COP \hspace{2 cm}$} & $i=5.25\% \textit{ ptv}$ \\
		\multicolumn{2}{|c|}{$\hspace{2 cm}n_1=10  \textit{ pav} \hspace{2 cm}$} & $g=5\% \textit{crecimiento geométrico periódico } $ \\
		\multicolumn{2}{|c|}{$\hspace{2 cm}n_2=2  \textit{ pav} \hspace{2 cm}$} & $R= ?COP $ \\ \hline	
		
		%%%%%%%%%% FIN DE MATEMÁTICAS
		%%%%% FIN DECLARACIÓN DE VARIABLES
		
		%%%%% INICIO FLUJO DE CAJA
		\rowcolor[HTML]{FFB183}
		\multicolumn{3}{|c|}{\cellcolor[HTML]{FFB183}\textbf{3. Diagrama de flujo de caja}} \\ \hline
		%Mezclamos 3 columnas y pondremos el dibujo
		%%%%%%%%%%%%% INSERCIÓN DE LA IMAGEN
		%Deberán descargar las imágenes respectivas del drive y pegarlas en la carpeta
		%n_capitulo/img/ejemplos/1/capitulo1ejemplo1.pdf  (el /1/ es el numero del ejemplo)
		\multicolumn{3}{|c|}{ \includegraphics[trim=-5 -5 -5 -5 , scale=0.8]{6_Capitulo/img/ejemplos/11/Capitulo6Ejemplo11.pdf} }
		\\ \hline
		%%%%%%%%%%%%% FIN INSERCIÓN DE IMAGEN
		%%%%%FIN FLUJO DE CAJA
		
		%%%%% INICIO DECLARACIÓN FORMULAS
		%%%%%%%%%%% INICIO TITULO
		\rowcolor[HTML]{FFB183}
		\multicolumn{3}{|c|}{\cellcolor[HTML]{FFB183}\textbf{4. Declaración de fórmulas}}    \\ \hline
		%%%%%%%%%%% FIN TITULO
		%%%%%%%%%%% INICIO MATEMÁTICAS
		\multicolumn{3}{|c|}{$VF=(\frac{(R)[(1+g)^{n}-(1+i)^{n}]}{g-i}) \hspace{0.4 cm} \textit{Valor futuro gradiente si } g \neq i $} \\  
		\multicolumn{3}{|c|}{$F=P(1+i)^{n} \hspace{0.4 cm} \textit{Valor futuro}$} \\ \hline
		
		%%%%%%%%%% FIN MATEMÁTICAS
		%%%%%% INICIO DESARROLLO MATEMÁTICO
		\rowcolor[HTML]{FFB183}
		%%%%%%%%%%INICIO TITULO
		\multicolumn{3}{|c|}{\cellcolor[HTML]{FFB183}\textbf{5. Desarrollo matemático}}       \\ \hline
		%%%%%%%%%% FIN TITULO
		%%%%%%%%%% INICIO MATEMÁTICAS
		\multicolumn{3}{|c|}{$ 500{.}000COP=\frac{(R)[(1-0.05)^{10}-(1+0.0525)^{-10}]}{0.5-0.0525} \cdot (1+0.0525)^{2}  \hspace{0.4 cm} $} \\ \hline
		%%%%%%%%%% FIN MATEMÁTICAS
		%%%%%% FIN DESARROLLO MATEMÁTICO
		%%%%%% INICIO RESPUESTA
		\rowcolor[HTML]{FFB183}
		%%%%%%%%%%INICIO TITULO
		\multicolumn{3}{|c|}{\cellcolor[HTML]{FFB183}\textbf{6. Respuesta}}   \\ \hline
		%%%%%%%%%% FIN TITULO
		%%%%%%%%%% INICIO RESPUESTA MATEMÁTICA
		\multicolumn{3}{|c|}{${R=  72{.}468.80COP}$} \\ \hline
		%%%%%%%%%% FIN MATEMÁTICAS
		%%%%%% FIN RESPUESTA
	\end{longtable}
	%Se crean dos lineas en blanco para que no quede el siguiente texto tan pegado
	%\newline \newline %USARLO SI CREES QUE ES NECESARIO
\end{center}

%%%%%%%%%%%%%%%%%%%%%%%%%%FIN EJERCICIO 11 %%%%%%%%%%%%%%%%%%%%%%%%%%%
