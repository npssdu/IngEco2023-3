%%%%%%%%%%%%%%%%%%% EJERCICIO 9.c  %%%%%%
\textbf{Parte c.} Una tasa periódica bimensual vencida. \\
%\newpage %USAR SOLO SI EL SOLUCIÓN QUEDA SOLO Y ES NECESARIO BAJARLO A LA SIGUIENTE PAGINA
\textbf{Solución c.}\\
%La tabla ira centrada
\begin{center}
   \renewcommand{\arraystretch}{1.5}% Margenes de las celdas
   %Creación de la cuadricula de 3 columnas
   \begin{longtable}[H]{|C{0.3\linewidth}|C{0.3\linewidth}|C{0.3\linewidth}|}
      %Creamos una linea horizontal
      \hline
      %Definimos el color de la primera fila
      \rowcolor[HTML]{FFB183}
      %%%%% INICIO ASIGNACIÓN PERíODO FOCAL %%%%%%%
      %%%%%%%%%% INICIO TITULO
      %Lo que se hace aquí es mezclar las 3 columnas en una sola
      \multicolumn{3}{|c|}{\cellcolor[HTML]{FFB183}\textbf{1. Asignación período focal}}                                                                                                 \\ \hline
      \multicolumn{3}{|c|}{$pf= \textit{No aplica}$}                                                                                                                                    
      \\ \hline
      %%%%%%%%%% FIN TITULO
      %%%%% INICIO DECLARACIÓN DE VARIABLES %%%%%%%
      %%%%%%%%%% INICIO TITULO
      %Lo que se hace aquí es mezclar las 3 columnas en una sola
      \multicolumn{3}{|c|}{\cellcolor[HTML]{FFB183}\textbf{2. Declaración de variables}}                                                                                                 \\ \hline
      %%%%%%%%%% FIN TITULO
      %%%%%%%%%% INICIO DE MATEMÁTICAS
      %Cada & hace referencia al paso de la siguiente columna

      $j_{1} = 24\% \textit{ namv}$          & $m_{1} = 12  \textit{ pmv}$                            & $i_{2} = ?\% \textit{ pbv} $ \\
      $i_{1}= 2\% \textit{ pmv}$       & $m_{2} = 2 \textit{ pbv} $                                                      &   $j_{2} = ?\% \textit{ nasv} $                             \\                                                            &                                                       &                               \\ \hline




      
      %%%%%%%%%% FIN DE MATEMÁTICAS
      %%%%% FIN DECLARACIÓN DE VARIABLES

      %%%%% INICIO FLUJO DE CAJA
      \rowcolor[HTML]{FFB183}
      \multicolumn{3}{|c|}{\cellcolor[HTML]{FFB183}\textbf{3. Diagrama de equivalencia de tasas}}                                                                                        \\ \hline
      %Mezclamos 3 columnas y pondremos el dibujo
      %%%%%%%%%%%%% INSERCIÓN DE LA IMAGEN
      %Deberán descargar las imágenes respectivas del drive y pegarlas en la carpeta
      %n_capitulo/img/ejemplos/1/capitulo1ejemplo1.pdf  (el /1/ es el numero del ejemplo)

      \multicolumn{3}{|c|}{ \includegraphics[trim=-5 -5 -5 -5 , scale=0.4]
      {2_Capitulo/img/ejemplos/6/Capitulo2Ejemplo6.pdf} } 
       \\ \hline
      
      %%%%%%%%%%%%% FIN INSERCIÓN DE IMAGEN
      %%%%%FIN FLUJO DE CAJA

      %%%%% INICIO DECLARACIÓN FORMULAS
      %%%%%%%%%%% INICIO TITULO
      \rowcolor[HTML]{FFB183}
      \multicolumn{3}{|c|}{\cellcolor[HTML]{FFB183}\textbf{4. Declaración de fórmulas}}                                                                                                  \\ \hline
      %%%%%%%%%%% FIN TITULO
      %%%%%%%%%%% INICIO MATEMÁTICAS

      \multicolumn{2}{|c|}{$(1+i_{1})^{m_{1}}=(1+i_{2})^{m_{2}} \textit{ Equivalencia de tasas}$} & $j_{2}=i_{2}\cdot m_{2} \textit{ Tasa nominal anual}$                                \\ \hline
      %%%%%%%%%% FIN MATEMÁTICAS
      %%%%%% INICIO DESARROLLO MATEMÁTICO
      \rowcolor[HTML]{FFB183}
      %%%%%%%%%%INICIO TITULO
      \multicolumn{3}{|c|}{\cellcolor[HTML]{FFB183}\textbf{5. Desarrollo matemático}}                                                                                                    \\ \hline
      %%%%%%%%%% FIN TITULO
      %%%%%%%%%% INICIO MATEMÁTICAS
      \multicolumn{2}{|c|}{$(1 + 0,02\%)^{12}= (1 + i_{2})^{6} $}                                 & {$j_{2}=(4,04\% \textit{ pbv})(6\textit{ pbv})$}                                                    \\
      \multicolumn{2}{|c|}{$(1,02)^{2}-1=i_{2}$ }                                      & {$j_{2} = 24,24\% \textit{ nabv}$}                                                                                        \\
      \multicolumn{2}{|c|}{$i_{2}= 0,0404 \textit{ pbv} \equiv 4,04\% \textit{ pbv}$}                                          &                                                                                                              \\ \hline
      %%%%%%%%%% FIN MATEMÁTICAS
      %%%%%% FIN DESARROLLO MATEMÁTICO
      %%%%%% INICIO RESPUESTA
      \rowcolor[HTML]{FFB183}
      %%%%%%%%%%INICIO TITULO
      \multicolumn{3}{|c|}{\cellcolor[HTML]{FFB183}\textbf{6. Respuesta}}                                                                                                                \\ \hline
      %%%%%%%%%% FIN TITULO
      %%%%%%%%%% INICIO RESPUESTA MATEMÁTICA
      \multicolumn{3}{|c|}{
         \begin{minipage}[t][0.03\textheight][c]{0.8\columnwidth}
            \centering
            ${j_{2} = 24,24\% nabv}$
            
         \end{minipage}
      }     

                                             \\ \hline
      %%%%%%%%%% FIN MATEMÁTICAS
      %%%%%% FIN RESPUESTA
   \end{longtable}
   %Se crean dos lineas en blanco para que no quede el siguiente texto tan pegado
   %\newline \newline %USARLO SI CREES QUE ES NECESARIO
\end{center}
%%%%%%%%%%%%%%%%%%% FIN EJERCICIO 9.c  %%%%%%%%%%%%%%%%%%%%%%%%%%%%%%
\newpage